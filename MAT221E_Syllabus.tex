% Options for packages loaded elsewhere
\PassOptionsToPackage{unicode}{hyperref}
\PassOptionsToPackage{hyphens}{url}
%
\documentclass[
  12pt,
]{article}
\usepackage{lmodern}
\usepackage{amsmath}
\usepackage{ifxetex,ifluatex}
\ifnum 0\ifxetex 1\fi\ifluatex 1\fi=0 % if pdftex
  \usepackage[T1]{fontenc}
  \usepackage[utf8]{inputenc}
  \usepackage{textcomp} % provide euro and other symbols
  \usepackage{amssymb}
\else % if luatex or xetex
  \usepackage{unicode-math}
  \defaultfontfeatures{Scale=MatchLowercase}
  \defaultfontfeatures[\rmfamily]{Ligatures=TeX,Scale=1}
\fi
% Use upquote if available, for straight quotes in verbatim environments
\IfFileExists{upquote.sty}{\usepackage{upquote}}{}
\IfFileExists{microtype.sty}{% use microtype if available
  \usepackage[]{microtype}
  \UseMicrotypeSet[protrusion]{basicmath} % disable protrusion for tt fonts
}{}
\makeatletter
\@ifundefined{KOMAClassName}{% if non-KOMA class
  \IfFileExists{parskip.sty}{%
    \usepackage{parskip}
  }{% else
    \setlength{\parindent}{0pt}
    \setlength{\parskip}{6pt plus 2pt minus 1pt}}
}{% if KOMA class
  \KOMAoptions{parskip=half}}
\makeatother
\usepackage{xcolor}
\IfFileExists{xurl.sty}{\usepackage{xurl}}{} % add URL line breaks if available
\IfFileExists{bookmark.sty}{\usepackage{bookmark}}{\usepackage{hyperref}}
\hypersetup{
  pdftitle={MAT221E: Probability Theory},
  hidelinks,
  pdfcreator={LaTeX via pandoc}}
\urlstyle{same} % disable monospaced font for URLs
\usepackage[margin=1in]{geometry}
\usepackage{graphicx}
\makeatletter
\def\maxwidth{\ifdim\Gin@nat@width>\linewidth\linewidth\else\Gin@nat@width\fi}
\def\maxheight{\ifdim\Gin@nat@height>\textheight\textheight\else\Gin@nat@height\fi}
\makeatother
% Scale images if necessary, so that they will not overflow the page
% margins by default, and it is still possible to overwrite the defaults
% using explicit options in \includegraphics[width, height, ...]{}
\setkeys{Gin}{width=\maxwidth,height=\maxheight,keepaspectratio}
% Set default figure placement to htbp
\makeatletter
\def\fps@figure{htbp}
\makeatother
\setlength{\emergencystretch}{3em} % prevent overfull lines
\providecommand{\tightlist}{%
  \setlength{\itemsep}{0pt}\setlength{\parskip}{0pt}}
\setcounter{secnumdepth}{-\maxdimen} % remove section numbering
\linespread{1.05} \usepackage{xcolor}
\ifluatex
  \usepackage{selnolig}  % disable illegal ligatures
\fi

\title{MAT221E: Probability Theory}
\author{}
\date{\vspace{-2.5em}Fall 2021}

\begin{document}
\maketitle

\hypertarget{instructor-information}{%
\subsection{Instructor Information}\label{instructor-information}}

\textbf{Instructor:} Gül İnan

\textbf{E-mail:}
\href{mailto:inan@itu.edu.tr}{\nolinkurl{inan@itu.edu.tr}}

\textbf{Office:} Room 424 @ Department of Mathematics

\textbf{Office hour:} You can ask me your questions right after the
class, or send me an e-mail for your queries and/or for scheduling an
online appointment via Zoom.

\hypertarget{course-information}{%
\subsection{Course Information}\label{course-information}}

\textbf{Course Type:} Must course for undergraduate students.

\textbf{Course Credits:} 3 local credits.

\textbf{Course Prerequisites:} None.

\textbf{Course Description:} This course is an introductory level
probability class on introducing following concepts: sample space,
probability measure on a sigma-algebra, Kolmogorov axioms, conditional
probability, combinatorial methods, Bayes theorem. Random variables,
discrete density functions, continuous density functions, functions of
random variables, bivariate joint density functions, marginal and
conditional density functions, independent random variables. Definition
and properties of expectations. Chebyshev inequality. Moment generating
functions. Special discrete and continuous distributions. Limit
theorems, law of large numbers, central limit theorem. Slutsky's
theorem.

\textbf{Class Schedule:} Thursdays between 11:30-14:30 p.m.

\textbf{Classroom:} Room D201 @ Faculty of Arts and Sciences.

\textbf{Covid-19 update} Wearing mask is required in shared spaces for
all individuals, including those who are fully vaccinated. For a full
list of regulations, please follow ITU main page.

\textbf{Course Objectives:} This course aims to:

\begin{enumerate}
\def\labelenumi{\arabic{enumi}.}
\tightlist
\item
  To provide the basic concepts of probability.
\item
  To set up probability models for a range of random phenomena, both
  discrete and continuous.
\item
  To develop critical thinking skills and abilities to apply calculus
  techniques (i.e., limits, derivatives, integration, infinite series)
  to assess the probability of an event.
\end{enumerate}

\textbf{Course Tentative Plan}: We will closely follow the weekly
schedule given below. However, weekly class schedules are subject to
change depending on the progress we make as a class.

Week 1. Experiments and events, algebra of events, sigma-algebra of
events, probability measure on a sigma-algebra, sigma algebra of borel
sets, Kolmogorov axioms, and finite sample spaces.

Week 2. Counting methods, combinatorial methods, product rule,
permutation, combination, binomial expansion, multinomial expansion,
tree diagram.

Week 3. Conditional probability, Bayes' rule, the law of total
probability, independence of events.

Week 4. Random variables and their distributions: Random variables,
distributions and probability mass functions, Bernoulli, Binomial,
Hypergeometric, and discrete uniform distributions.

Week 5. Cumulative distribution function, functions of random variables,
independence of random variables.

Week 6. Definition of Expectation, Special expectations: mean and
variance. Geometric, Negative Binomial, and Poisson distributions.

Week 7. Continuous random variables: Probability density functions,
Uniform, Normal and Exponential distributions.

Week 8. ITU Fall Break

Week 9. \textbf{Midterm on December 2, 2021 during lecture hour and
place.}

Week 10. Moments: Summaries of a distribution, sample moments, moment
generating functions.

Week 11. Joint distributions: Joint, marginal, and conditional
distributions. Covariance and correlation. Multinomial distribution.
Multivariate Normal distribution.

Week 12. Transformations: Change of variables, convolutions, Beta and
Gamma distributions.

Week 13. Marginal and conditional density functions, independent random
variables.

Week 14. Conditional expectation. Conditional variance.

Week 15. Inequalities. Limit theorems: law of large numbers and central
limit theorem. Slutsky's theorem. Chi-square, Student-t, and F
distributions.

\textbf{Student Learning Outcomes:} A student who completed this course
successfully is expected to:

\begin{enumerate}
\def\labelenumi{\arabic{enumi}.}
\tightlist
\item
  Understand and apply basic concepts of probability.
\item
  Understand probability distributions for both discrete and continuous
  phenomena.
\item
  Calculate basic characteristics such as mean and variance of
  probability distributions, and any probability associated with this
  distributions.
\item
  Use special probability distributions for modeling events.
\item
  Use limit theorems.
\end{enumerate}

immediately following the course, and/or a few months after the course.

\textbf{Textbook:} All lecture materials. Lecture materials (notes,
assignments, etc) will be uploaded on
\href{Ninova}{https://ninova.itu.edu.tr}.

\textbf{Course Workload:} 2 quizzes, 1 midterm exam, and 1 final exam
(see the grading policy below).

\textbf{Recommended Bibliography:} Students are encouraged to consult
the following sources on their own:

\begin{enumerate}
\def\labelenumi{\arabic{enumi}.}
\tightlist
\item
  Hogg, V.H. and Craig, A.T. (1995).
  \textit{Introduction to Mathematical Statistics}. New Jersey:
  Prentice-Hall International. {[}Hard copy available at ITU Mustafa
  Inan Library with CALL \#QA276 .H643 1995{]}. (Available on Ninova)
\item
  Hogg, R. V., Tanis, E. A., and Zimmerman, D.L. (2010).
  \textit{Probability and Statistical Inference}. Upper Saddle River,
  NJ, USA: Pearson/Prentice Hall. (Available on Ninova)
\item
  DeGroot., M.H. and Schervish, M.J. (2012).
  \textit{Probability and Statistics}. Boston: Addison-Wesley, c2012.
  {[}Hard copy available at ITU Mustafa Inan Library with CALL \#QA273
  .D445 2012{]}. (Available on Ninova)
\end{enumerate}

\textbf{Off-Campus Access to the ITU Library E-sources:}

Access to library e-sources remotely is possible with a library account.
Users without a library account should apply for the library
registration at
\href{Library register}{https://kutuphane.itu.edu.tr/en/register}. After
setting the web configurations given at
\href{Proxy}{https://kutuphane.itu.edu.tr/en/services \linebreak /web-browser-proxy-settings}
only once on your computer, you will able to have an access to ITU
Library e-sources.

\textbf{Selected Important Dates:} For the official ITU Fall 2021
academic calendar, please visit:

\url{https://www.sis.itu.edu.tr/TR/ogrenci/akademik-takvim/akademik-takvimler/takvim2022/lisans-akademik-takvimi.php}

Here are some selected important dates in Fall 2021 semester:

October 4, 2021: First day of classes.

October 4-8, 2021: Add-drop week.

October 29, 2021: Republic Day of Turkey (Friday, No classes).

November 22-26, 2021: ITU Fall Break (No classes).

January 1, 2022: New year (Saturday).

January 14, 2022: Last day of classes.

January 17-30, 2022: Final exam week.

I also honor other national and religious holidays. Students, who needs
flexibility on individual-based studies overlapping with these special
days, can inform me.

\hypertarget{course-policies}{%
\subsection{Course Policies}\label{course-policies}}

Please read the information below as a reference for how this class will
be conducted.

\textbf{Grading Policy:}

\begin{itemize}
\item
  Assessment Method \quad      \quad \quad                Total
  Contribution to Final Grade:

  \begin{itemize}
  \tightlist
  \item
    2 quizzes each 10\%,\\
  \item
    1 midterm exam 40\%,
  \item
    1 final exam 40\%.
  \end{itemize}
\item
  \textbf{Midterm date}: December 2, 2021 during the lecture hour in
  class.
\item
  Quizzes will be held during the lecture hours and the dates will be
  announced at well in advance. No make-up for quizzes.
\item
  \textbf{Student studies, namely, quizzes and exam papers which are not
  written well, does not follow a proper mathematical writing language,
  and are hard to review, will get ``0'' credit for that question.}
\item
  Please see an example for \textbf{a good homework} on Ninova.
\item
  Please read the general advice given at:

  \url{http://ma117.math.metu.edu.tr/course-info/general-advice/}.
\end{itemize}

\textbf{Late Submission Policy:} There are \textbf{NO} make-ups for
missed quizzes.

\textbf{Final Exam Attendance Policy:} At least 30 points from
in-semester studies (e.g., (Midterm * 0.4 + Quiz 1 * 0.1 + Quiz 2 * 0.1)
greater than or equal to 30)

\textbf{Make-Up Exam Policy:} The students who miss either midterm exam
or final exam due to a health problem can take a make-up exam as long as
they have a valid medical report taken on the exam day. The medical
report should be handed in immediately (within two days of its
expiration).

\textbf{Class Attendance Policy:}

The students are encouraged to attend classes, whereas the attendance is
not mandatory in Fall 2021 semester due to Covid-19 pandemic. For that
reason, the student is deemed responsible to manage his/her absences.

\textbf{Participation Policy:}

The students are expected to ask and answer questions, participate in
in-class activities, and show their interest and engagement in the
class.

\textbf{E-mail Policy:}

Please:

\begin{enumerate}
\def\labelenumi{\arabic{enumi}.}
\tightlist
\item
  Use a proper descriptive subject line (which may consist of the course
  number MAT221E followed by a short phrase summarizing the subject of
  your e-mail).
\item
  Start off your e-mail with a proper greeting, introduce yourself (give
  your name), then state your problem as short as possible.
\item
  Finally, use a proper closing and then finish your e-mail with your
  first name and so on.
\end{enumerate}

Feel free to send me e-mails. But be sure you that give me enough time
to get back to you. In the past, I have had pretty much tolerance for
e-mail messages sent after business hours and at weekends. But, now, due
to pandemic, I should say that I may not appreciate these e-mails
anymore. Lastly, e-mails asking for grade grubbing at the end of the
semester are not welcomed.

\textbf{Academic Honesty Policy:}

At every stage of the academic life, every ITU student is responsible
for obeying the academic honesty policy of ITU stated below:

\url{https://odek.itu.edu.tr/en/code-of-honor/ethics-in-university-life}.

\textbf{Equity, Diversity, and Inclusion:}

In this class, I am committed to cultural and individual differences and
diversity as including, but not limited to, age, disability, ethnicity,
gender, gender identity, language, national origin, race, religion,
culture, and socioeconomic status and I acknowledge the value of
differences.

\textbf{Student with Special Needs:}

If you are a student with special needs, let me know that how we can
adjust the course environment and materials in accordance with your
needs. Furthermore, you are also invited to contact the office of
students with special needs at:

\url{http://engelsiz.itu.edu.tr/}.

\end{document}
